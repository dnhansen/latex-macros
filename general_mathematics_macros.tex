\RequirePackage{mathtools}
\RequirePackage{amssymb}
\RequirePackage{bm}
\RequirePackage{suffix}

%%%%%%%%%%%
%% To do &&
%%%%%%%%%%%

% Mixed partial derivatives.
% Function definitions.
% Delimiters abs, norm, inner product automatically add a cdot if no argument is specified. (See mathtools documentation.)
% Vector space specific inner products. (See mathtools documentation.)
% Automatic spacing around differentials (like that defined by physics package).
% Consider generalising \Ltwo to general L^p spaces.
% Identity and zero matrices. And consider the future of \identity!


%%%%%%%%%%%
%% Notes %%
%%%%%%%%%%%

% In case of alignment issues or other problems with vectors, see https://tex.stackexchange.com/a/10643/63353.
% Note that '\span' is a primitive command used in \multicolumn (see https://tex.stackexchange.com/questions/33264/span-as-a-math-operator), so the span of a set of vectors is defined as '\Span'.
% L^2: Explicit domains should probably be specified manually, since they in general require parentheses (for instance L^2(R)), but not in the case of intervals. But consider using an if-statement to check.


%%%%%%%%%%%%%%%%%%%%%
%% Vector notation %%
%%%%%%%%%%%%%%%%%%%%%

% Typeset bold upright vectors. Use starred version for greek letters and symbols. The argument to bm contains an extra pair of braces to avoid alignment problems.
\renewcommand{\vec}[1]{\mathbf{#1}}
\WithSuffix\newcommand\vec*[1]{\bm{{#1}}}

% Alternative bold italic vectors. Keep the starred version for compatibility with the above version.
%\renewcommand{\vec}[1]{\bm{{#1}}}
%\WithSuffix\newcommand\vec*[1]{\bm{{#1}}}

% Hat vectors
\newcommand{\hatvec}[1]{\hat{\vec{#1}}}
\WithSuffix\newcommand\hatvec*[1]{\hat{\vec*{{#1}}}}

% Unit vectors
\newcommand{\ihat}{\bm{\hat{\textbf{\i}}}}
\newcommand{\jhat}{\bm{\hat{\textbf{\j}}}}
\newcommand{\khat}{\hatvec{k}}
\newcommand{\xhat}{\hatvec{x}}
\newcommand{\yhat}{\hatvec{y}}
\newcommand{\zhat}{\hatvec{z}}


%%%%%%%%%%%%%%%%%%%%
%% Linear algebra %%
%%%%%%%%%%%%%%%%%%%%

% Maps, operators and matrices
\newcommand{\op}[1]{\hat{#1}}					% Operator
\newcommand{\identity}{\op{1}}					% Identity operator
\renewcommand{\det}{\operatorname{det}}			% Determinant
\DeclareMathOperator{\tr}{Tr}					% Trace
\newcommand{\trans}[1]{#1^{T}}					% Matrix transpose

% Vector spaces
\renewcommand{\dim}{\operatorname{dim}}			% Dimension
\DeclareMathOperator{\Span}{span}				% Span (see note)
\newcommand{\orthog}[1]{#1^{\perp}}				% Orthogonal complement

% Maps on vector spaces
\DeclarePairedDelimiterX{\inner}[2]%
	{\langle}{\rangle}{#1, #2}					% Inner product
\DeclarePairedDelimiter{\norm}{\lVert}{\rVert}	% Norm


%%%%%%%%%%%%%%
%% Calculus %%
%%%%%%%%%%%%%%

% Derivatives
\newcommand{\dif}{\mathrm{d}}					% Differential
\newcommand{\dv}[3][\hspace{-0.5pt}]%
	{\frac{\dif^{#1} #2}{\dif #3^{#1}}}			% Ordinary derivative
\newcommand{\pdv}[3][\hspace{-0.5pt}]%
	{\frac{\partial^{#1} #2}{\partial #3^{#1}}}	% Partial derivative

% Vector operators
\let\div\divsymb
\newcommand{\grad}[1]{\nabla #1}				% Gradient
\newcommand{\div}[1]{\nabla \cdot #1}			% Divergence
\newcommand{\curl}[1]{\nabla \times #1}			% Curl


%%%%%%%%%%
%% Sets %%
%%%%%%%%%%

% Set builder notation
\DeclarePairedDelimiterX{\set}[2]{\{}{\}}{#1\,\delimsize\vert\,\mathopen{}#2}

% Miscellaneous sets
\newcommand{\continuous}[3][\hspace{-0.5pt}]%
	{C^{#1} (#2,#3)}							% Continuously differentiable functions
\newcommand{\smooth}[2]{C^\infty (#1,#2)}		% Smooth functions
\newcommand{\linear}[2]{\mathcal{L}(#1,#2)}		% Linear maps
\newcommand{\Hilbert}{\mathcal{H}}				% Hilbert space
\newcommand{\Ltwo}{L^2}							% L^2 (see notes)
\newcommand{\Mat}[2]{\mathrm{Mat}_{#1}(#2)}		% Set Mat_n(F) of matrices

% Blackboard bold sets
\newcommand{\N}{\mathbb{N}}						% Natural numbers
\newcommand{\Z}{\mathbb{Z}}						% Integers
\newcommand{\Q}{\mathbb{Q}}						% Rational numbers
\newcommand{\R}{\mathbb{R}}						% Real numbers
\newcommand{\C}{\mathbb{C}}						% Complex numbers
\newcommand{\K}{\mathbb{K}}						% K field
\newcommand{\F}{\mathbb{F}}						% F field

% Intervals
\DeclarePairedDelimiter{\intoo}{]}{[}			% Open
\DeclarePairedDelimiter{\intcc}{[}{]}			% Closed
\DeclarePairedDelimiter{\intoc}{]}{]}			% Open left, closed right
\DeclarePairedDelimiter{\intco}{[}{[}			% Closed right, open left


%%%%%%%%%%%%%%%%%%%
%% Miscellaneous %%
%%%%%%%%%%%%%%%%%%%

% Complex numbers
\renewcommand{\Re}{\operatorname{Re}}			% Real part
\renewcommand{\Im}{\operatorname{Im}}			% Imaginary part

% Maps
\renewcommand{\ker}{\operatorname{ker}}			% Kernel
\DeclareMathOperator{\im}{Im}					% Image
\newcommand{\id}[1][\hspace{-0.5pt}]%
	{\mathrm{Id}_{#1}}							% Identity map

% Miscellaneous operations
\newcommand{\mean}[1]{\overline{#1}}			% Mean
\DeclarePairedDelimiter{\abs}{\lvert}{\rvert}	% Absolute value
\DeclareMathOperator{\sgn}{sgn}					% Sign
\renewcommand{\deg}{\operatorname{deg}}			% Degree (e.g. of polynomium)

% Statistics
\newcommand{\std}[1][\hspace{0.7pt}]%
	{\sigma_{#1}}								% Standard deviation
\newcommand{\var}{\operatorname{Var}}			% Variance


