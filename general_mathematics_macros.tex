% =====================================================
% REQUIRED PACKAGES
% =====================================================

\RequirePackage{mathtools}
\RequirePackage{amssymb}
\RequirePackage{bm}
\RequirePackage{suffix}
\RequirePackage{etoolbox}



% =====================================================
% TO DO
% =====================================================

%  *  A few words on how to use the commands in general.
%
%  *  Mixed partial derivatives. (Long term project!)
%
%  *  Function definitions. (Probably too many moving parts for it to be easier with a dedicated command.)
%
%  *  Consider generalising \Ltwo to general L^p spaces.
%
%  *  Consider redefining intervals to take two arguments.
%
%  *  More convenient way to make roman subscripts (and super-?).
%
%  *  Differentials: Find automatic way to check if the next symbol is a full stop or comma, in which case the spacing should be diminished. Until then, making the punctuation part of the argument solves the problem (but it's a hack).
%
%  *  Consider separating \inner out into physics and mathematics specific files. Especially since the definition is now more complex. Or define the delimiter globally elsewhere.
%
%  *  Conditional probabilities.
%
%  *  Automatically check whether a vector is Latin or not. (Seems a bit complicated to do in general, may rely on xparse.)
%
%  *  Infinite sequence: Automatically detect the (maybe?) last letter of the input and start it from that letter = 1 by default. (Is this maybe too computationally expensive?)
%
%  *  DeclarePairedDelimiterXPP: Can it take optional arguments? I.e., is it really necessary to define auxilliary commands? It really shouldn't be!



% =====================================================
% Notes
% =====================================================

%  *  This script redefines the following commands:
%		\deg
%		\d		(redefined as \underdot)
%		\det
%		\dim
%		\div	(redefined as \divsymb)
%		\Im
%		\ker
%		\Re
%		\vec
%
%  *  In case of alignment issues or other problems with vectors, see https://tex.stackexchange.com/a/10643/63353.
%
%  *  Note that '\span' is a primitive command used in \multicolumn (see https://tex.stackexchange.com/questions/33264/span-as-a-math-operator), so the span of a set of vectors is defined as '\Span'.
%
%  *  L^2: Explicit domains should probably be specified manually, since they in general require parentheses (for instance L^2(R)), but not in the case of intervals. But consider using an if-statement to check.

%%%%%%%%%%%%%%%%%%%%%%%%%%%%%%%%%%%%%%%%%%%%%%%%%%%%%%%
%%%%%%%%%%%%%%%%%%%%% FRACTIONS %%%%%%%%%%%%%%%%%%%%%%%
%%%%%%%%%%%%%%%%%%%%%%%%%%%%%%%%%%%%%%%%%%%%%%%%%%%%%%%

% Define fraction command which automatically detects inline math.
\newcommand{\myfrac}[2]{
	\mathchoice
	{\frac{#1}{#2}}     % \displaystyle
	{#1/#2}             % \textstyle
	{\frac{#1}{#2}}     % \scriptstyle
	{\frac{#1}{#2}}     % \scriptscriptstyle
}

% See also: https://tex.stackexchange.com/a/223561/63353




%%%%%%%%%%%%%%%%%%%%%%%%%%%%%%%%%%%%%%%%%%%%%%%%%%%%%%%
%%%%%%%%%%%%%%%%%% VECTOR NOTATION %%%%%%%%%%%%%%%%%%%%
%%%%%%%%%%%%%%%%%%%%%%%%%%%%%%%%%%%%%%%%%%%%%%%%%%%%%%%

% =====================================================
% GENERAL VECTORS
% =====================================================

% Typeset bold upright vectors. Use starred version for greek letters and symbols. The argument to bm contains an extra pair of braces to avoid alignment problems.
\renewcommand{\vec}[1]{\mathbf{#1}}
\WithSuffix\newcommand\vec*[1]{\bm{{#1}}}

% Alternative bold italic vectors. Keep the starred version for compatibility with the above version (maybe?).
%\renewcommand{\vec}[1]{\bm{{#1}}}
%\WithSuffix\newcommand\vec*[1]{\bm{{#1}}}



% =====================================================
% SPECIAL VECTORS
% =====================================================

% Hat vectors
\newcommand{\hatvec}[1]{\hat{\vec{#1}}}
\WithSuffix\newcommand\hatvec*[1]{\hat{\vec*{{#1}}}}

% Unit vectors
\newcommand{\ihat}{\bm{\hat{\textbf{\i}}}}
\newcommand{\jhat}{\bm{\hat{\textbf{\j}}}}
\newcommand{\khat}{\hatvec{k}}
\newcommand{\xhat}{\hatvec{x}}
\newcommand{\yhat}{\hatvec{y}}
\newcommand{\zhat}{\hatvec{z}}




%%%%%%%%%%%%%%%%%%%%%%%%%%%%%%%%%%%%%%%%%%%%%%%%%%%%%%%
%%%%%%%%%%%%%%%%%%% LINEAR ALGEBRA %%%%%%%%%%%%%%%%%%%%
%%%%%%%%%%%%%%%%%%%%%%%%%%%%%%%%%%%%%%%%%%%%%%%%%%%%%%%

% =====================================================
% GENERAL
% =====================================================

% Maps, operators and matrices
\newcommand{\op}[1]{\hat{#1}}					% Operator
\newcommand{\identity}{\op{1}}					% Identity operator
\renewcommand{\det}{\operatorname{det}}			% Determinant
\DeclareMathOperator{\tr}{Tr}					% Trace
\newcommand{\trans}[1]{#1^{T}}					% Matrix transpose

% Vector spaces
\renewcommand{\dim}{\operatorname{dim}}			% Dimension
\DeclareMathOperator{\Span}{span}				% Span (see note)
\newcommand{\orthog}[1]{#1^{\perp}}				% Orthogonal complement



% =====================================================
% INNER PRODUCT
% =====================================================

% General inner product
\DeclarePairedDelimiterX{\innerNoPostCode}[2]{\langle}{\rangle}
	{
		\ifblank{#1}{\:\cdot\:}{#1},
		\ifblank{#2}{\:\cdot\:}{#2}
	}

% With respect to a particular vector space (as post code)
\DeclarePairedDelimiterXPP{\innerPostCode}[3]{}{\langle}{\rangle}{_{#1}}
	{
		\ifblank{#2}{\:\cdot\:}{#2},
		\ifblank{#3}{\:\cdot\:}{#3}
	}

% Single command with optional argument to specify particular vector space
\newcommand{\inner}[3][]{
	\ifblank{#1}{
		\innerNoPostCode{#2}{#3}
	}{
		\innerPostCode{#1}{#2}{#3}
	}
}

% Starred version (automatic resizing)
\WithSuffix\newcommand\inner*[3][]{
	\ifblank{#1}{
		\innerNoPostCode*{#2}{#3}
	}{
		\innerPostCode*{#1}{#2}{#3}
	}
}



% =====================================================
% NORM
% =====================================================

% General norm
\DeclarePairedDelimiterX{\normNoPostCode}[1]{\lVert}{\rVert}
	{
		\ifblank{#1}{\:\cdot\:}{#1}
	}

% With respect to a particular vector space (as post code)
\DeclarePairedDelimiterXPP{\normPostCode}[2]{}{\lVert}{\rVert}{_{#1}}
	{
		\ifblank{#2}{\:\cdot\:}{#2}
	}

% Single command with optional argument to specify particular vector space
\newcommand{\norm}[2][]{
	\ifblank{#1}{
		\normNoPostCode{#2}
	}{
		\normPostCode{#1}{#2}
	}
}

% Starred version (automatic resizing)
\WithSuffix\newcommand\norm*[2][]{
	\ifblank{#1}{
		\normNoPostCode*{#2}
	}{
		\normPostCode*{#1}{#2}
	}
}




%%%%%%%%%%%%%%%%%%%%%%%%%%%%%%%%%%%%%%%%%%%%%%%%%%%%%%%
%%%%%%%%%%%%%%%%%%%%%% CALCULUS %%%%%%%%%%%%%%%%%%%%%%%
%%%%%%%%%%%%%%%%%%%%%%%%%%%%%%%%%%%%%%%%%%%%%%%%%%%%%%%

% =====================================================
% ONE DIMENSION
% =====================================================

% Differential
\let\d\underdot
\newcommand{\d}[2][\hspace{-0.5pt}]{\mathop{\mathrm{d}^{#1}#2}}

% Partial differential
\newcommand{\pd}[2][\hspace{-0.5pt}]{\mathop{\partial^{#1}#2}}

% Ordinary derivative
\newcommand{\dv}[3][\hspace{-0.5pt}]{\myfrac{\d[#1]{#2}}{\d{#3}^{#1}}}

% Partial derivative
\newcommand{\pdv}[3][\hspace{-0.5pt}]{\myfrac{\pd[#1]{#2}}{\pd{#3}^{#1}}}



% =====================================================
% MULTI-DIMENSIONAL
% =====================================================

% Vector operators: To differentiate with respect to a particular vector, use the optional argument.

% Gradient
\newcommand{\grad}[2][\hspace{0.5pt}]{\nabla_{\!#1} #2}

% Divergence
\let\div\divsymb
\newcommand{\div}[2][\hspace{0.5pt}]{\nabla_{\!#1} \cdot #2}

% Curl
\newcommand{\curl}[2][\hspace{0.5pt}]{\nabla_{\!#1} \times #2}




%%%%%%%%%%%%%%%%%%%%%%%%%%%%%%%%%%%%%%%%%%%%%%%%%%%%%%%
%%%%%%%%%%%%%%%%%%%%%%%%% SETS %%%%%%%%%%%%%%%%%%%%%%%%
%%%%%%%%%%%%%%%%%%%%%%%%%%%%%%%%%%%%%%%%%%%%%%%%%%%%%%%

% Set builder notation
\DeclarePairedDelimiterX{\set}[2]{\{}{\}}{#1\,\delimsize\vert\,\mathopen{}#2}

% Miscellaneous sets
\newcommand{\continuous}[3][\hspace{-0.5pt}]%
	{C^{#1} (#2,#3)}							% Continuously differentiable functions
\newcommand{\smooth}[2]{C^\infty (#1,#2)}		% Smooth functions
\newcommand{\linear}[2]{\mathcal{L}(#1,#2)}		% Linear maps
\newcommand{\Hilbert}{\mathcal{H}}				% Hilbert space
\newcommand{\Ltwo}{L^2}							% L^2 (see notes)
\newcommand{\Mat}[2]{\mathrm{Mat}_{#1}(#2)}		% Set Mat_n(F) of matrices

% Blackboard bold sets
\newcommand{\N}{\mathbb{N}}						% Natural numbers
\newcommand{\Z}{\mathbb{Z}}						% Integers
\newcommand{\Q}{\mathbb{Q}}						% Rational numbers
\newcommand{\R}{\mathbb{R}}						% Real numbers
\newcommand{\C}{\mathbb{C}}						% Complex numbers
\newcommand{\K}{\mathbb{K}}						% K field
\newcommand{\F}{\mathbb{F}}						% F field

% Intervals
\DeclarePairedDelimiter{\intoo}{]}{[}			% Open
\DeclarePairedDelimiter{\intcc}{[}{]}			% Closed
\DeclarePairedDelimiter{\intoc}{]}{]}			% Open left, closed right
\DeclarePairedDelimiter{\intco}{[}{[}			% Closed right, open left




%%%%%%%%%%%%%%%%%%%%%%%%%%%%%%%%%%%%%%%%%%%%%%%%%%%%%%%
%%%%%%%%%%%%%%%%%%%% MISCELLANEOUS %%%%%%%%%%%%%%%%%%%%
%%%%%%%%%%%%%%%%%%%%%%%%%%%%%%%%%%%%%%%%%%%%%%%%%%%%%%%

% =====================================================
% OPERATIONS
% =====================================================

% Mean
\newcommand{\mean}[1]{\overline{#1}}

% Absolute value
\DeclarePairedDelimiterX{\abs}[1]{\lvert}{\rvert}{
	\ifblank{#1}{\:\cdot\:}{#1}
}

% Sign
\DeclareMathOperator{\sgn}{sgn}

% Degree (e.g. of polynomial)
\renewcommand{\deg}{\operatorname{deg}}



% =====================================================
% INFINITE SEQUENCES
% =====================================================

% Define delimiters
\newcommand{\seqLeftDelimiter}{\{}
\newcommand{\seqRightDelimiter}{\}}

% Alternative delimiters
%\newcommand{\seqLeftDelimiter}{(}
%\newcommand{\seqRightDelimiter}{)}

% Define sequence as paired delimiter
\DeclarePairedDelimiterXPP{\infiniteSequence}[2]{}{\seqLeftDelimiter}{\seqRightDelimiter}{_{#2}^{\infty}}{#1}

% New command with optional argument to specify lower bound
\newcommand{\seq}[2][n=1]{\infiniteSequence{#2}{#1}}

% Starred version (automatic resizing)
\WithSuffix\newcommand\seq*[2][n=1]{\infiniteSequence*{#2}{#1}}



% =====================================================
% MISCELLANEOUS
% =====================================================

% Complex numbers
\renewcommand{\Re}{\operatorname{Re}}			% Real part
\renewcommand{\Im}{\operatorname{Im}}			% Imaginary part

% Maps
\renewcommand{\ker}{\operatorname{ker}}			% Kernel
\DeclareMathOperator{\im}{Im}					% Image
\newcommand{\id}[1][\hspace{-0.5pt}]%
	{\mathrm{Id}_{#1}}							% Identity map

% Statistics
\newcommand{\std}[1][\hspace{0.7pt}]%
	{\sigma_{#1}}								% Standard deviation
\newcommand{\var}{\operatorname{Var}}			% Variance






