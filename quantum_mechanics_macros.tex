\RequirePackage{mathtools}
\RequirePackage{etoolbox}
\RequirePackage{suffix}

%%%%%%%%%%%
%% To do %%
%%%%%%%%%%%

% Subscript after expectation value. Is it necessary to rewrite the command, or is "_sub" sufficient?


%%%%%%%%%%%%%%%%%%%%%%
%% Bra-ket notation %%
%%%%%%%%%%%%%%%%%%%%%%

% Bra. Use starred versions for automatic resizing.
\newcommand{\bra}[1]{\mathinner {\langle {#1}\rvert}}
\WithSuffix\newcommand\bra*[1]{\mathinner {\left\langle {#1}\right|}}

% Ket
\newcommand{\ket}[1]{\mathinner {\lvert {#1}\rangle}}
\WithSuffix\newcommand\ket*[1]{\mathinner {\left| {#1}\right\rangle}}

% Bra-ket. Takes two arguments.
\DeclarePairedDelimiterX{\braket}[2]%
{\langle}{\rangle}{
	\ifblank{#1}{\:\cdot\:}{#1}
	\,\delimsize\vert\,\mathopen{}
	\ifblank{#2}{\:\cdot\:}{#2}
}

% Matrix elements. Takes three arguments.
\DeclarePairedDelimiterX{\matrixel}[3]{\langle}{\rangle}%
	{#1\,\delimsize\vert\,\mathopen{}#2\,\delimsize\vert\,\mathopen{}#3}

% Commutator. Takes two arguments.
\DeclarePairedDelimiterX{\comm}[2]{[}{]}{#1, #2}

% Expectation value. If no optional argument is specified, it returns the ordinary expectation value '<Q>', but if the state 'x' is specified as an optional argument, it instead returns the matrix element '<x|Q|x>'.
\DeclarePairedDelimiter{\expectationvalue}{\langle}{\rangle}
\newcommand{\expval}[2][]{%
	\ifblank{#1}%
	{\expectationvalue{%
		\ifblank{#2}{\:\cdot\:}{#2}
	}}									% If no optional argument
	{\matrixel{#1}{#2}{#1}}				% If optional argument
}
\WithSuffix\newcommand\expval*[2][]{%	% Starred version
	\ifblank{#1}%
	{\expectationvalue*{%
			\ifblank{#2}{\:\cdot\:}{#2}
	}}									% If no optional argument
	{\matrixel*{#1}{#2}{#1}}				% If optional argument
}

